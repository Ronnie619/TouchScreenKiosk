\documentclass[]{article}

%opening
\title{Meeting Notes for TouchStone}
\author{Ronnie Rodriguez, Joseph Higley, Morgan Holbart}

\begin{document}

\maketitle

\section{Meeting Notes for Tuesday 9/15/2105}
\begin{itemize}
	\item Meeting went from 2:30pm - 4:30pm.
	\item All team members were present for this meeting.
		\item Meeting began by catching Morgan up on what he missed at the team formation meeting the previous Thursday.
		\item Met with the Dean of Engineering to go over exactly what he wants the product to accomplish:
			\begin{itemize}
				\item Must be cutting edge. It needs to really speak to the fact that you are entering the engineering college. It needs to be something that brings people to the building to see, like the Think Tank does.
				\item  Needs to provide a visually appealing, easy to understand interface that users can get important information from, such as campus maps, instructor information, past Deans, etc.
				\item Needs to somehow honor Dean Janssen, the father of the lady who is funding this project. Possibly incorporated into the name?
				\item Needs to be within a budget of roughly \$6,000 to \$10,000.
			\end{itemize}
		\item After the Dean left, we brainstormed ideas for the remainder of the time:
			\begin{itemize}
				\item What kind of app?
					\begin{itemize}
						\item Web-based? (HTML, CSS, JS, Python/PHP)
						\item Dedicated App coded in Java or C++?
					\end{itemize}
				\item What technology could we use?
					\begin{itemize}
						\item Projector with touch kiosk?
						\item Virtual Reality?
						\item Projector with Kinect to enable touch features?
					\end{itemize}
			\end{itemize}		
		\item Action items for next meeting:
			\begin{itemize}
				\item All: Research possible directions to take the project.
				\item All: Get Logbooks for recording important information throughout the semester.
			\end{itemize}
	\end{itemize}

\section{Meeting Notes for Tuesday 9/22/2015}
	\begin{itemize}
		\item Meeting went from 2:30pm-3:30pm
		\item All team members were present for this meeting.
	\end{itemize}
	\begin{itemize}
		\item Meeting began by discussing with Morgan our reservations with having a VR component in the project. 
			\begin{itemize}
				\item Virtual Reality tech is very new and not fully proven.
				\item A VR component would completely dash any sort of tour-giving angle of the project, which is something the Dean stated he wanted for the kiosk. 
				\item A VR headset(s) on a stand won't have any sort of WOW factor upon entering the college of engineering, another thing the Dean said was critical for the project.
				\item There technically isn't even a usable, consumer level VR headset available for use, with the earliest one coming next year (2016). 
			\end{itemize}
		\item Morgan demonstrated some of the tech, however, and it was persusive. We decided to run the idea by 
		the dean at our next meeting, and see if he thinks it would be sufficient for what he's looking for. 
		\item Joe and Ronnie were in agreement that we should pursue a different angle.
			\begin{itemize}
				\item A massive (8-10 foot) touchscreen that enables users to access all the information laid out in the form given to us by the Dean. 
					\begin{itemize}
						\item This would have a significant WOW factor. 
						\item With the use of projectors and glass panes, this could easily be done within the budget alotted.
						\item This would be ideal for a campus tour type situation where one person is presenting to a group. 
						\item An interactive touchscreen of that size definitely fits the criteria of something that would attract people to the building ala the Think Tank. 
					\end{itemize}
			\end{itemize}
		\item Action items for this sprint:
			\begin{itemize}
				\item Morgan: Continue pursuing the VR angle and see what it would take to get Unity licenses for the group in the event that we go that route.
				\item Joe: Research viable projectors that could be used for the project in the event we go that route.
				\item Ronnie: Research techs that can make a projected image an interactive touchscreen.
			\end{itemize}
	\end{itemize}
	
\section{Meeting Notes for Tuesday 9/29/2015}
\begin{itemize}
	\item Meeting went from 2:30 - 3:30
	\item Morgan Holbart was absent from this meeting.
	\item After discussing it with Bruce, we, as a team, have decided to pursue the large touchscreen angle for this project. The VR is perhaps something that can be integrated down the road another semester. 
	\item We will, however, still use Unity coupled with C\# as our primary development framework for the project. Unity3D is ideal for our purposes, because it streamlines the process of making attractive and responsive user interfaces, as it is primarily used for creating video games. 
	\item Action items for previous sprint:
		\begin{itemize}
			\item Morgan: Continue pursuing the VR angle and see what it would take to get Unity licenses for the group in the event that we go that route.
			\item Joe: Research viable projectors that could be used for the project in the event we go that route.
			\item Ronnie: Research techs that can make a projected image an interactive touchscreen
		\end{itemize}
	\item What was accomplished?
		\begin{itemize}
			\item Morgan continued looking into what it would take to obtain a Unity3D license for the group project. It seems that we would indeed have to get the commerical version of the software, as the University is technically a money making institution. So, we're looking at spending around \$1400 for a commercial license. 
			\item Joe found a fantastic short-throw projector that would be ideal for our purposes. 
				\begin{itemize}
					\item Casio XJ-UT310WN
						\begin{itemize}
							\item The XJ-UT310WN can project bright, shadow-free images up to 110” diagonal. The built-in short throw lens and advanced mirror system provides an extremely close throw ratio (0.28:1) and enables you to project an 80” image from just 1.5 feet away.
							\item This projector will enable us to utilize as little outward space possible in the hallway. 
							\item Cost: Approximately \$1,600
						\end{itemize}
				\end{itemize}
			\item Ronnie found a technology called UBI Interactive, which is a software coupled with a proprietary camera device that essentially turns any surface into a touchscreen by registering the touchpoints of users on the surface. This tech works with Windows 8 and Windows 10. It seems to be exactly the kind of thing we need for this project. More info is currently needed before committing to purchase, though.
		\end{itemize}
		\item Action items for next sprint:
			\begin{itemize}
				\item All: Start watching Unity3D tutorials and brushing up on C\#.
				\item Ronnie: Email UBI interactive and see about setting up a Skype meeting to have a live demonstration of the software.
				\item Joe: Start some preliminary mock-ups of some possible user interface design elements. 
			\end{itemize}
\end{itemize}

\section{Meeting Notes for Tuesday 10/19/2015}
\begin{itemize}
	\item Meeting went from 2:30 - 3:30
	\item Morgan was absent for this meeting.
	\item Action items for previous sprint:
		\begin{itemize}
			\item All: Start watching Unity3D tutorials and brushing up on C\#.
			\item Ronnie: Email UBI interactive and see about setting up a Skype meeting to have a live demonstration of the software.
			\item Joe: Start some preliminary mock-ups of some possible user interface design elements. 
		\end{itemize}
	\item What was accomplished?
		\begin{itemize}
			\item Joe finished some user interface mockups. He demonstrated a potential tile-focused design that looked very nice. 
			\item Ronnie successfully set up a skype meeting with the CEO of UBI Interactive, scheduled for 2:30 on 10/19/2015. Ronnie requested that the team be presented with how UBI handles certain things that will be needed for the project such as basic menu navigation, maps, scrolling, etc. 
			\item Morgan worked on and demonstrated a rough example of how we could split the screen into separate modules in Unity3D. 
		\end{itemize}
	\item Meeting focused almost entirely on Joe and Ronnie having a conversation with the co-founder of UBI Interactive. 	
	\item As expected, UBI Interactive seems to be exactly the kind of thing we're looking for on this project. The demonstration went very well. UBI is fast, responsive, and easy to set-up and use. 
	\item After the conversation, Joe showed the team exactly where the screen would be installed, which he learned from his meeting with the Dean. It will be on the eastern wall, because this wall already has convenient rear access via a service closet. 
	\item Action items for next sprint:
		\begin{itemize}
			\item Morgan and Joe: Research and potentially code an example of the best possible tile layout and the maximum number of users that can be supported on the kiosk.
			\item Ronnie: Research the possibility of incorporating an interactive maps component into the kiosk ala Google Maps.
		\end{itemize}
\end{itemize}

\section{Meeting Notes for Tuesday 11/2/2015}
\begin{itemize}
	\item Meeting went from 2:30 - 3:30
	\item All team members were present for this meeting.
	\item Joe updated us on the status of the Wiki. It's going well.
	\item Action items for the previous sprint:
		\begin{itemize}
			\item Morgan and Joe: Research and potentially code an example of the best possible tile layout and the maximum number of users that can be supported on the kiosk.
			\item Ronnie: Research the possibility of incorporating an interactive maps component into the kiosk ala Google Maps.
		\end{itemize}
	\item What was accomplished?
		\begin{itemize}
			\item Joe and Morgan did further research into the best way to split up the tiles and the best number of people to allow to use the kiosk at one time.
			\item Ronnie did further research into integrating an interactive maps component into the kiosk.
		\end{itemize}
	\item Joe demonstrated some 3D menu ideas he's been toying around with. 
	\item We talked about how to best layout the screen as far as how many tiles, and after toying with several ideas, decided it would be best to optimize the number and layout of the tiles to support up to 3 users. The base menu for each user would have nine tiles each approximately a foot across for one user, and the size of each tile would scale down from that for each additional user, so each user can also have nine base tiles. 
	\item Ronnie discussed that he had found a free Unity3D asset that incorporates static images from Google Maps into a Unity project, but realized this isn't exactly right for our vision. We need the maps to be interactive. 
	\item Action items for next sprint:
		\begin{itemize}
			 \item Joe: Come up with some more potential art design mock-ups. 
			 \item Ronnie: Do further research into the possibility of integrating Google Maps into our app. If a sufficient asset does not currently exist, research what would be necessary to roll our own asset from scratch. 
			 \item Morgan: Put more work into splitting the screen into multiple parts for multiple users. Something demonstrable would be ideal, as it would help us a lot at the next snapshot.
		\end{itemize} 
\end{itemize}
	
	\section{Meeting Notes for Tuesday 11/17/2015}
	\begin{itemize}
		\item Meeting Time:
		\begin{itemize}
			\item 2:30pm - 3:45pm
		\end{itemize}
		\item Goals for the Previous Sprint:
		\begin{itemize}
			\item Morgan: Research how we can split the user interface effectively into 3 separate, but fully functional entities.
			\item Ronnie: Research what would be necessary to incorporate an interactive Google Maps component into the project. 
			\item Joe: Work on some potential art designs for how the app may eventually look. 
		\end{itemize}
		\item What was accomplished?
		\begin{itemize}
			\item Morgan: Completed and demonstrated a rough demo showing it would be a fairly simple task to make it so the interface could be divided into three separately functioning modules. The source code was put on the github repo. Morgan also demonstrated a side project he had worked on, which can best be described as an easy-to-use graphical Unity interface builder that, when completed, will make it so down the line, when someone (faculty, deans, etc.) want to update their information, they can do so very easily without messing up the way the app functions. 
			\item Ronnie: Discovered an already built and well-reviewed asset in the Unity store that incorporates an interactable Google Maps into an object. This asset is ready to go off-the-shelf, and would be a massive time saver versus rolling an asset ourselves. The asset is a steal at 60 dollars, and the team has agreed that we should purchase it and incorporate it into our project. 
			\item Joe: Made some mock-ups of potential user interface elements that incorporate images, titles, and symbols. Joe also worked on the wiki for our project. 
		\end{itemize}
			\item What was discussed at the meeting?
		\begin{itemize}
			\item The three of us demonstrated the things we accomplished during the last sprint (as laid out above).
			\item We met with Larry Stauffer.
				\begin{itemize}
					\item Larry mostly wanted to discuss his desire to implement speakers into the kiosk. 
					\item What kind of speakers? (Studio monitors, computer speakers, surround sound, etc.)
		  			\item How many speakers?
					\item Where would the speakers be placed?
				\end{itemize}
		\end{itemize}
			 \item What needs to be accomplished in the next sprint?
				 \begin{itemize}
				 	\item Morgan: Continue working on the graphical interface builder.
				 	\item Ronnie: Begin research and/or start preliminary coding of web scraping and info serialization for the app. 
				 	\item Joe: Make more art mock-ups and get the ball rolling on buying everything we need to get started with actual development. 
				 \end{itemize}
	\end{itemize}

\section{Meeting Notes for Tuesday 12/1/2015}
\begin{itemize}
	\item Meeting went from 2:30pm-3:30pm
	\item All team members were present for this meeting.
	\item Action items for previous sprint:
		\begin{itemize}
			\item Morgan: Continue working on the graphical interface builder.
			\item Ronnie: Begin research and/or start preliminary coding of web scraping and info serialization for the app. 
			\item Joe: Make more art mock-ups and get the ball rolling on buying everything we need to get started with actual development.
		\end{itemize}
	\item What was accomplished?
	\begin{itemize}
		\item Morgan made a great amount of progress on the UI builder. It will be of great value for our demonstration at snapshot.
		\item Ronnie did a large amount of research on how we are to get the information we need from the web. 
			\begin{itemize}
				\item Web Scraping would be ideal for most of the things we want need to display, since most of that information is already on the university's various websites in one form or another. The issue there, however, is that web scraping tends to rely on well-formatted HTML code that adheres to industry standards. This is not the case with the University's code. A great portion of it is incredibly malformed, which would make it impossible to scrape by normal methods. Thankfully, however, Ronnie discovered a web scraping framework for .NET code called HTML Agility Pack.
				\item HTML Agility Pack is an agile HTML parser that builds a read/write DOM and supports plain XPATH or XSLT (you actually don't HAVE to understand XPATH nor XSLT to use it, don't worry...). It is a .NET code library that allows you to parse "out of the web" HTML files. The parser is very tolerant with "real world" malformed HTML. The object model is very similar to what proposes System.Xml, but for HTML documents (or streams). This is exactly what we need to get the info from the sites.
				\item There are some bits of information that we will have to build ourselves, however, as they are not currently compiled on any websites. For instance, information and pictures of past Deans of Engineering.
			\end{itemize}
		\item Joe made a good amount of progress on some more potential art designs for the project that incorporate symbols as well as words to describe to the user what each tile does. His examples also have a function that causes the background image of each tile to intermittently change so as to keep things fresh for the user. Joe also got the ball rolling on ordering the things needed for us to move on to the next stage of development, which included the UBI Interactive software as well as the short-throw projector.
	\end{itemize}
	\item Action items for next sprint:
		\begin{itemize}
			\item All: Get everything done so far together for our snapshot presentation.
		\end{itemize}
\end{itemize}
\end{document}


